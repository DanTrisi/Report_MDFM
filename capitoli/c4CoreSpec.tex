To do so, we built a small dataset collecting with other manufacturers data and then, using a simple linear polynomial fit, we interpolated it with our dimensions in order to have our splitting force target.

\begin{longtable}{@{}p{3.5cm}p{3.3cm}M{3.3cm}P{3.5cm}@{}}
\caption{Manufacturers data used to interpolate our splitting force target} \\
\toprule
\textbf{Manufacturer} & \textbf{Maximum log diameter (mm)} & \textbf{Maximum log length (mm)} & \textbf{Maximum splitting force (kN)} \\
\midrule
\endfirsthead

\multicolumn{4}{c}{{\bfseries Table \thetable\ (continued)}} \\
\toprule
\textbf{N°} & \textbf{Product Design Specification} & \textbf{Value} & \textbf{R/D} \\
\midrule
\endhead

\bottomrule
\multicolumn{4}{r}{{Continued on next page}} \\
\endfoot

\bottomrule
\endlastfoot

% ====== 8 example rows to fill in ======
Rabaud & 800 & 500 & 143 \\[1pt]
Varimatic & 300 & 550 & 62.3 \\[1pt]
Brugger & 500 & 600 & 267 \\[1pt]
FuelWood & 800 & 1000 & 151 \\[1pt]
Bizon-Ins & 380 & 550 & 142 \\[1pt]
Amix & 450 & 550 & 133 \\[1pt]


\end{longtable}

\begin{figure}[H]
    \centering
    \begin{minipage}[b]{0.7\textwidth}
        \centering
        \includegraphics[width=\textwidth]{Immagini/Splitting_force.png}
        \caption{Collected data from other manufacturers to evaluate our splitting force}
        \label{fig:splitforce}
    \end{minipage}
    \hfill
\end{figure}

Based on that, and using the following code in \textbf{RStudio}

\lstset{
  basicstyle=\ttfamily\small,
  keywordstyle=\color{blue},
  commentstyle=\color{gray},
  stringstyle=\color{red},
  breaklines=true,
  frame=single
}

\begin{lstlisting}[language=R]
data.lm <- lm(split_force ~ poly(max_d, 1, raw = T) + poly(max_l, 1, raw = T), data = df)
summary(data.lm)
df %>% 
  add_predictions(data.lm)

our_chunk <- tibble(
  max_d = 700,
  max_l = 250
)
predict(data.lm, newdata = our_chunk)
\end{lstlisting}

our splitting force resulted in

\textbf{Splitting force} = 166 kN

